\section{Motivation}

Civil aviation has primarily been focused on efficiently moving goods and people via the airspace to their destination. The application of sound engineering practices and conservative operating procedures has made flying to be  the safest form of transport. The desire not to compromise such a safe system has made if difficult to introduce unmanned and autonomous vehicles into the airspace. The rules for operating aircraft in the US national airspace are typically relaxed when needed to support natural disaster relief such as wildfire fighting and hurricane relief. NASA Aeronautics' Airspace Operations and Safety Program (AOSP) System Wide Safety (SWS) project has been using  has been investigating how unmanned and manned aircraft may safely operate in shared airspace and using wildfire fighting and hurricane relief as use cases.


 Traditionally  civil aviation has employed  simple communication patterns between air and ground and among aircraft. The typical airborne communication is very simple in that aircraft broadcast their position information via automatic dependent surveillance-broadcast (ADSB). Air-ground communication is basically radio communication from pilots to air traffic controllers.  The use cases under consideration demand more sophisticated coordination sometimes between airborne and ground based elements in order to accomplish shared mission goals such as to extinguish fire. Yet the operating environment will not admit the use of reliable high-bandwidth internet connections that enable reliable transfers of large data. At best we can assume an ad-hoc wireless mesh network facilitating communication  between aircraft as well as  between aircraft and ground crews.  We cannot assume a reliable communication medium. Ground crews on different sides of a ridge line will almost certainly loose direct communication with each other and if the ridge line is high enough the ground crew may lose communication with low flying aircraft on the other side. Ad-Hoc networks may allow messages to be routed around such barriers, but at some cost of additional communications. 
