\section{Desiderata for distributed applications}
\label{sec:des}

The CAP theorem, and others like it, place fundamental limitations on
the consistency of real-world distributed systems. In the absense of a
``perfect'' system, engineers are forced to make tradeoffs. Ideally,
these tradeoffs should be tuned for the specific application in
mind---a protocol that works well in a datacenter might not work well
in a heterogeneous geodistributed setting. This section lists three
desirable features of distributed systems and frameworks for reasoning
about or implementing them. We chose this set based on the particular
details of civil aviation and disaster response, where safety is a
high priority and usage/communication patterns may be
unpredictable.

These desiderata can be understood narrowly for the specific
application of data replication, but these same considerations can be
applied to any distributed application, such as the sheaf-theoretic
data fusion in Section \ref{sec:sheaf}.

\subsubsection*{D1: Quantifiable bounds on inconsistency}

\emph{A distributed application should quantify the amount of
consistency it delivers. That is, it should (1) provide a mathematical
way of measuring inconsistency between system nodes, and (2) bound
this value while the system is available.}

The CAP theorem implies that an available data replication application
cannot bound inconsistency in all circumstances. When bounded
inconsistency cannot be guaranteed, a system satisfying D1 may become
unavailable. Alternatively, a reasonable behavior would be to continue
providing some form of availability, but alert the user that due to
network and system use conditions the requisite level of consisteny
cannot be guaranteed by the application, leaving the user with the
choice to assess the risk and continue using the system with a weaker
inconsistency bound.

\subsubsection*{D2: Accommodation of heterogeneous nodes}

\emph{An application should not assume that there is a typical system
node. Instead, the system should accomdate a diverse range of
heterogeneous clients presenting different capabilities, tasks, and
risk-factors.}

One can expect a variety of hardware in the field. For example,
wildfires often involve responses from many different fire
departments, and it must be assumed that they are not always using
identical systems. Different participants in the system may be solving
different tasks, with different levels of access to the network, and
they present different risks. With these sorts of factors in mind, one
should hope for frameworks that are as general as possible to
accomodate a wide variety of clients.

\subsubsection*{D3: Optimization for a geodistributed wide area network}

\emph{An application should be optimized for the sorts of
communication patterns that occur in geodistributed wide area networks
(WANs) under real-world conditions.}

Consider two incidents. Wouldn't want to enforce needless global
consistency, particularly if the agents in one area do not have the
same consistency requirements for another area.

Network throughput has some (perhaps approximately linear)
relationship with throughput. Communications patterns are likely far
from uniform too. In fact, these two things likely coincide---it is
often that nodes which are nearby have a stronger need to coordinate
their actions than nodes which are far away. For example, consider
manoeauvering airplanes to avoid crash.
