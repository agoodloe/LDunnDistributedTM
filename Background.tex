\section{Background}
A distributed system is a collection of independent entities that cooperate to
solve a problem that cannot be individually solved
\cite{kshemkalyani_singhal_2008}. Or in a bit more specificity, a distributed
system is
\begin{quotation}
   A collection of computers that do not share common memory or a common
   physical clock, that communicate by a messages passing over a communication
   network, and where each computer has its own memory and runs its own
   operating system. Typically the computers are semi-autonomous and are loosely
   coupled while they cooperate to address a problem collectively.
   \cite{10.5555/562065}
\end{quotation}
Compared to a standalone computer, distributed systems introduce fundamental
challenges in both theory and practice. To discuss the matter, we fix some
terminology common to the systems literature. Hereafter, a \emph{system} is
always distributed. Individual participants in the system are its
\emph{nodes}---the set of nodes may change as the system evolves in time. A
\emph{correct} node is one whose hardware is functioning as designed and which
faithfully implements the protocol or coordination scheme under consideration.
Any other node is \emph{failing}---perhaps for innocent reasons like hardware
failure, or perhaps because the node is an adversary.\footnote{A certain class
of innocent failures, now termed \emph{Byzantine faults} and known to occur in
the wild, can exhibit behavior that looks like that of an intentionally
deceptive adversary. \cite{1979Sift}} System nodes typically receive and handle
requests from clients, often human users, often to read or write some value in a
shared database or to reserve some limited resource like a synchronization
primitive or airline ticket. Another kind of node might be one attached to
\emph{sensors}, such as a weather-monitoring satellite, in which case the goal
is often to integrate the global set of sensor data and distribute this data to
interested clients.

A basic challenge of a distributed system is to present the abstraction of a
single global node shared by all clients, i.e. to hide the distributed nature of
the system. For instance, suppose nodes $A$ and $B$ cooperate to maintain a
shared value $n \in \mathbb{N}$ which can be accessed from either node. This
requires the nodes to pass messages to each other, used to notify the other of
updates. A client who issues a write request on node $A$ probably wishes to
immediately observe the effects of this update upon issuing a read request to
node $B$. Note that a client may not have any control, knowledge, or interest in
which node is serving any particular request---they expect $A$ and $B$ to behave
identically.\footnote{Though a system may be implemented with a predictable
client-node pairing scheme, such as a content-distribution network (CDN) which
routes web requests to the caching node geographically nearest to the client.
Such a design exploits \emph{locality}, a major theme of this paper we get into
later.} That a multiple-node system is fundamentally more complex than an
individual node is evidenced by the fact that the previous goal is not
realizable in a certain extremal sense.

\begin{theorem}[Brewer's CAP Theorem]
   A distributed system cannot achieve all three of the following conditions at
   the same time.
   \begin{enumerate}
      \item \textbf{Consistency}: Clients always read the value of the most
        recent write (possibly issued to another node by another
        client).\footnote{Formally, Gilbert and Lynch \cite{2002gilbertlynchCAP}
        take consistency to mean \emph{linearizability}.}
      \item \textbf{Availability}: A client request is always eventually served
        (possibly after an unbounded but finite amount of time).
      \item \textbf{Partition-tolerance}: The system operates even in a
        situation where network communication links between nodes are
        unavailable.
   \end{enumerate}
\end{theorem}
\begin{proof}
   It is clear why this is called the CAP theorem, and we will sometimes use the
   letters C, A, and P to refer to these properties. The theorem is generally
   attributed to Brewer \cite{2000brewerCAP}, though the first totally rigorous
   proof was given in Gilbert and Lynch \cite{2002gilbertlynchCAP}. At the
   informal level of this report, the proof is easy. Continuing our example from
   earlier, suppose both $A$ and $B$ have the shared understanding $n = k - 1$
   when a network partition occurs and separates the nodes. We do not assume the
   network will ever recover---perhaps a router has failed and will not be
   replaced.





   Now suppose a client issues a write update $n \textrm{ := } k$ to node $A$.
   $A$ must eventually accept this update, at which point all future read
   requests to $A$ will return $k$, as otherwise $A$ would be unavailable. Now
   suppose a client issues a read request to $B$. $B$ has not seen the update at
   node $A$, but to maintain availability it must eventually return the stale
   value $k-1$, which violates consistency---$A$ and $B$ no longer present the
   illusion of a single node.
\end{proof}

In the proof of CAP, it may seem extreme not to assume that the network will
become available again, not even after an unbounded amount of time. Indeed,
several aspects of the theorem are highly idealized. Note for example that
``availability'' only means eventually, with no bounds on the amount of time the
client may need to wait. Nonetheless, the theorem highlights a deeper phenomenon
that occurs throughout the study of distributed systems, namely a fundamental
tension between ensuring both liveness and safety properties when the network is
not completely reliable.\footnote{Broadening our perspective, one may see this
as a fundamental tension in logic between soundness (a safety property) and
completeness (a kind of liveness), such as embodied in G\"odel's famed
incompleteness theorems.}  \cite{2012perspectivesCAP} \emph{Liveness} is the
property that a system will eventually do something good, like respond to a
client. \emph{Safety} is the property that a system never does something bad,
like respond in an observably inconsistent way. In short, the only way to
guarantee safety in the presence of network disruption requires doing nothing at
all, comprising liveness. Said from another angle, if all nodes in a system
resolve to continue functioning despite network disruption, there is always a
chance something bad will happen.

Despite this inherent tradeoff, the framing of CAP as a ``choose 2-of-3''
problem is misleading, as claimed by no less an authority than the author:

\begin{quote}
   The ``2 of 3" formulation was always misleading because it tended to
   oversimplify the tensions among properties. Now such nuances matter. CAP
   prohibits only a tiny part of the design space: perfect availability and
   consistency in the presence of partitions, which are rare.
   \cite{@2012brewerCAPchanged}
\end{quote}

In the context of this quote, Brewer had in mind such systems as one might find
in a modern datacenter, where total network disruption is likely to be uncommon.
Bewer also had in mind a decade's worth of research on detecting and working
around network disruption, such as on the modern internet where multiple routes
between two nodes are typically available. In such settings,  heavy-duty
disruptions like a total outage \cite{2021facebookBGP} are exceptional events
and smaller partitions have a limited impact on overall system performance.

In the context of wildfire fighting and other disaster relief, it makes sense to
assume that partitions are not rare,
or even to assume that some amount of network partitioning is expected under
normal conditions. Thinking of extreme scenarios such as those considered in the
Solar System Internet \cite{2016nasaSSI}, one can imagine situations where the
ability to pass messages between two nodes is a more exceptional scenario than a
network partition. Despite this difference of context, the point of Brewer's
message is the same: in the presence of a disruptive network, the engineering
challenge is to strike a reasonable balance between consistency and
availability, not simply to choose one to the exclusion of the other.

So far we have discussed ``distributed systems'' in the abstract. In practice,
systems come in a variety of different kinds, ranging from certain kinds of
loosely coupled multiprocessor designs all the way up to a sophisticated
worldwide network of datacenters. Of course, the exact details of the
system---its scale, purpose, and network characteristics, among other
things---will influence precisely how we go about balancing its properties and
building applications. With some of the basic difficulties now outlined, we turn
to a consideration of the defining characteristics of the systems considered in
wildfire  fighting and disaster releif.

