\section{Continuous Consistency}
\label{sec:contcons}

We have discussed consistency as discrete proposition: a distributed
application provides strong consistency or it does not. In many
real-world scenarios, it makes sense to work with data that is
consistent up to some $\epsilon > 0$. For instance, it may be
acceptable if a web cache is 30 minutes out of date, but not 48 hours
out of date. By weakening our consistency requirements to some
tolerable numeric level, we can imagine applications that provide
relatively good consistency and availability. This would offer neither
perfect C or A according to the CAP theorem. It also would not offer
P, as such bounds cannot be guaranteed in the presense of indefinite
network partitions. However, it may be robust relative to the sorts of
network behavior encountered in practice.

Yu and Vahdat explored the CAP tradeoff from this perspective in a
series of papers \cite{2000tact} \cite{2000tactalgorithms}
\cite{10.5555/1251229.1251250} \cite{DBLP:conf/icdcs/YuV01}
\cite{2002tact}. They propose a theory of \emph{conits}, roughly a
unit of data subject to their theory of continuous consistency. By
controlling the threshold of acceptable inconsistency on each conit as
a continuous quantity, applications can precisely control the
observable tradeoff between consistency and performance, allowing
trading one for the other in a gradual fashion. They built a prototype
toolkit called TACT, which allows applications to specify precisely
their desired levels of consistency for each conit.

An interesting aspect of this work is that consistency can be tuned
\emph{dynamically}. This is desirable because one does not know a
priori how much consistency or availability is acceptable. Indeed, the
ideal tradeoff will depend on the precise network performance and
communication patterns in the field, both of which can be expected to
change in time. Say, because new network endpoints are added, or
because some scenario leads to an unusually high level of
communication, straining the network.

\subsection{Consistency metrics for conits}

In the TACT framework, each conit is associated with three independent
metrics for measuring consistency: order error, real-time staleness,
and numeric consistency. Each conit may set the inconsistency bounds
for these metrics independently, allowing for fine-grained control.

\paragraph{Numeric consistency}

Give a simplified explanation of the algorithm for split-weight
absolute error bounding. Lorem ipsum dolor sit amet, consectetur
adipiscing elit. Etiam et ex nisl. Integer hendrerit ante
purus. Mauris bibendum neque vitae nibh tristique, at vestibulum neque
efficitur. Nam ut quam in purus venenatis interdum. Morbi nec velit et
ipsum congue hendrerit ut vitae nibh. Nunc vel augue nulla. Mauris eu
dolor lorem. Suspendisse sapien justo, dapibus mollis eleifend
euismod, pretium vitae est.

Praesent et tincidunt justo. Aenean consectetur est eu rutrum
mollis. Aliquam aliquam ante vel magna vehicula pellentesque. In
tristique convallis felis, et lobortis nisl cursus at. Nunc semper
purus augue, sit amet interdum tortor pretium quis. Integer quis dui
ac magna euismod maximus. Proin euismod neque nisl, lobortis auctor
purus euismod ac. Nam et lorem feugiat mauris ultricies convallis in
eu metus. Vivamus sollicitudin nisi scelerisque, vehicula erat id,
mollis enim. Sed in odio ut nibh tincidunt faucibus.

\subsubsection*{Real-time staleness}

Lamport vector clock. Lorem ipsum dolor sit amet, consectetur
adipiscing elit. Etiam et ex nisl. Integer hendrerit ante
purus. Mauris bibendum neque vitae nibh tristique, at vestibulum neque
efficitur. Nam ut quam in purus venenatis interdum. Morbi nec velit et
ipsum congue hendrerit ut vitae nibh. Nunc vel augue nulla. Mauris eu
dolor lorem. Suspendisse sapien justo, dapibus mollis eleifend
euismod, pretium vitae est.

Praesent et tincidunt justo. Aenean consectetur est eu rutrum
mollis. Aliquam aliquam ante vel magna vehicula pellentesque. In
tristique convallis felis, et lobortis nisl cursus at. Nunc semper
purus augue, sit amet interdum tortor pretium quis. Integer quis dui
ac magna euismod maximus. Proin euismod neque nisl, lobortis auctor
purus euismod ac. Nam et lorem feugiat mauris ultricies convallis in
eu metus. Vivamus sollicitudin nisi scelerisque, vehicula erat id,
mollis enim. Sed in odio ut nibh tincidunt faucibus.

\subsubsection*{Commit order error}

Requires a commitment protocol. Lorem ipsum dolor sit amet,
consectetur adipiscing elit. Etiam et ex nisl. Integer hendrerit ante
purus. Mauris bibendum neque vitae nibh tristique, at vestibulum neque
efficitur. Nam ut quam in purus venenatis interdum. Morbi nec velit et
ipsum congue hendrerit ut vitae nibh. Nunc vel augue nulla. Mauris eu
dolor lorem. Suspendisse sapien justo, dapibus mollis eleifend
euismod, pretium vitae est.

Praesent et tincidunt justo. Aenean consectetur est eu rutrum
mollis. Aliquam aliquam ante vel magna vehicula pellentesque. In
tristique convallis felis, et lobortis nisl cursus at. Nunc semper
purus augue, sit amet interdum tortor pretium quis. Integer quis dui
ac magna euismod maximus. Proin euismod neque nisl, lobortis auctor
purus euismod ac. Nam et lorem feugiat mauris ultricies convallis in
eu metus. Vivamus sollicitudin nisi scelerisque, vehicula erat id,
mollis enim. Sed in odio ut nibh tincidunt faucibus.

\subsection{Applications of conits}

Lorem ipsum dolor sit amet, consectetur adipiscing elit. Etiam et ex
nisl. Integer hendrerit ante purus. Mauris bibendum neque vitae nibh
tristique, at vestibulum neque efficitur. Nam ut quam in purus
venenatis interdum. Morbi nec velit et ipsum congue hendrerit ut vitae
nibh. Nunc vel augue nulla. Mauris eu dolor lorem. Suspendisse sapien
justo, dapibus mollis eleifend euismod, pretium vitae est.

Praesent et tincidunt justo. Aenean consectetur est eu rutrum
mollis. Aliquam aliquam ante vel magna vehicula pellentesque. In
tristique convallis felis, et lobortis nisl cursus at. Nunc semper
purus augue, sit amet interdum tortor pretium quis. Integer quis dui
ac magna euismod maximus. Proin euismod neque nisl, lobortis auctor
purus euismod ac. Nam et lorem feugiat mauris ultricies convallis in
eu metus. Vivamus sollicitudin nisi scelerisque, vehicula erat id,
mollis enim. Sed in odio ut nibh tincidunt faucibus.  Lorem ipsum
dolor sit amet, consectetur adipiscing elit. Etiam et ex nisl. Integer
hendrerit ante purus. Mauris bibendum neque vitae nibh tristique, at
vestibulum neque efficitur. Nam ut quam in purus venenatis
interdum. Morbi nec velit et ipsum congue hendrerit ut vitae
nibh. Nunc vel augue nulla. Mauris eu dolor lorem. Suspendisse sapien
justo, dapibus mollis eleifend euismod, pretium vitae est.

Praesent et tincidunt justo. Aenean consectetur est eu rutrum
mollis. Aliquam aliquam ante vel magna vehicula pellentesque. In
tristique convallis felis, et lobortis nisl cursus at. Nunc semper
purus augue, sit amet interdum tortor pretium quis. Integer quis dui
ac magna euismod maximus. Proin euismod neque nisl, lobortis auctor
purus euismod ac. Nam et lorem feugiat mauris ultricies convallis in
eu metus. Vivamus sollicitudin nisi scelerisque, vehicula erat id,
mollis enim. Sed in odio ut nibh tincidunt faucibus.

\subsection{Evaluation of conits}

- By providing three metrics for (in)consistency between conits, and
  ensuring that these metrics always fall within bounds set by the
  user, TACT clearly satisfies our requirement that a framework should
  provide quantified, bounded consistency.

- The ability to dynamically set values supports heterogeneity
  D2. Some nodes may require highly consistent information, while
  others may prioritize availability. In many cases one can imagine
  that the exact tradeoff favored by a node will be a function of the
  network conditions and usage patterns, which cannot easily be
  predicted in advance. Therefore, the ability to set these values on
  a per-node basis, and to update them dynamically, satisfies our
  requirement that a framework supports a wide variety of heterogenous
  node, rather than optimizing for a simple case in which all nodes
  are assumed to share certain traits in common.

- Less clear is how this framework may be optimized for the particular
  usage patterns that occur in a geodistributed network. The
  developers of TACT-like replication protocols have several
  parameters to choose from, such as the exact protocols for
  disseminating updates and commiting writes in a total order. As we
  discuss as Future Work, the selection of these protocols should be
  selected based on real-world experiments, and may require further
  theoretical investigations.

\subsection{Future work}

TACT is a prototype, not something that would function now. It should
be taken as inspiration for a next-generation continuous consistency
replication framework. However, developing such an application for the
exact scenarios we envision would require theoretical refine and
real-world investigations.

- What network optimizations can be done? This is an empirical
  question that would require real-world tests. Additionally, a numer
  of protocol parameters---e.g., what commitment protocol for
  writes. Gossip vs anti-entropy. Likely would need to carefully
  consider how these protocols interact with the routing layer of the
  network.
- Additionally, TACT was not designed for a scenario where nodes may
  rapidly enter or exit the system. An unmodified version of TACT
  would likely perform badly in response to node crashes---without
  being able to detect and work around such adverse events, writes
  could never be committed without violating all three conit metrics.
  Detecting and recovering from such a scenario is a critical
  question.
- TACT is based on the assumption that all parties are replicating a
  global database. It makes sense for a conit to be a single-data
  item, and then you only have to replicate that conit, rather than
  the entire database. It makes sense that there is a large global
  database, and that individual systems have "views" of the data,
  rather than each node maintaining a whole copy. As a simple example,
  suppose each system node requires a detailed visualization of a
  fire's trajectory within. A central database maintains up-to-date
  about all fires---how can these parties collaborate? Due to network
  limitations, wouldn't make sense to replicate the entire
  database. This could lead to "overlapping" conits.
