\section{Introduction}
\label{sec:introduction}

Civil aviation has traditionally focused primarily on the efficient
and safe transportation of people and goods via the airspace. Owing to
the application of sound engineering practices and conservative
operating procedures, flying is today the safest mode of
transport. The desire not to compromise this safety makes it
challenging to introduce uncrewed vehicles into the airspace. To that
end, the NASA Aeronautics' Airspace Operations and Safety Program
(AOSP) System Wide Safety (SWS) project has been investigating how
crewed and uncrewed aircraft may safely operate in shared
airspace. Wildfire suppression and hurricane relief are being studied
as motivating use cases, in part because the rules for operating in
the US national airspace are typically relaxed during natural
disasters and relief efforts.

Disaster response scenarios present unique challenges for safe
operations. One major challenge is the need for system-wide
coordination in spite of an unpredictable communications
environment. Traditionally, civil aviation has employed simple
communication patterns between airborne and ground-based agents and
among aircraft. For instance, aircraft equipped with Automatic
Dependent Surveillance-Broadcast (ADS-B) monitor their location using
GPS and periodically broadcast this information to air traffic
controllers and nearby aircraft. The use cases under consideration
demand more sophisticated coordination schemes between airborne and
ground-based elements to collectively accomplish mission goals like
navigating safely in close proximity, delivering resources to remote
locations, and suppressing fires.

Unfortunately, the operating environment will not admit the use of
reliable, high-throughput internet connections that consistently allow
clients to transmit large amounts of data to each other quickly. For
instance, obstructions like distance, terrain, smoke, and weather mean
we should expect network packets to be dropped or delayed in
unpredictable ways. We also expect the network characteristics to vary
between deployments and to evolve dynamically in time. These factors
make network performance difficult to predict and control, which in
turn makes it difficult to coordinate distributed agents and offer
strong safety guarantees.

Designing systems that are resilient to these sorts of environments is
a challenge for distributed computing, a subdiscipline of computer
science. This purpose of this memorandum is to enumerate some of the
considerations involved in coordinating air- and ground-based elements
from a distributed computing perspective, identifying challenges,
potential requirements, and frameworks that may aid in developing
solutions.

Our central theme is to emphasize \emph{continuous} notions of
consistency between distributed agents. We shall see that na\"ive
models of consistency provide strong safety guarantees but are
unworkably brittle. For instance, Brewer's CAP theorem (Theorem
\ref{thm:cap}) implies that \emph{atomic} consistency cannot be
enforced by a highly-available system in the presense of network
partitions---which are themselves guaranteed in real-world
conditions. Therefore, we concentrate on notions of consistency that
vary smoothly and quantifiably, which should result in more general
frameworks that can provide safety guarantees even in real-world
scenarious.

\subsection{Layout of this document}

This document aims to be self-contained and readable to a broad
technical audience. It is laid out as follows.

Section \ref{sec:distrsys} provides a high-level introduction to
distributed systems and consistency models. We define two ``strong''
models, \emph{atomic} and \emph{sequential} consistency, both of which
provide highly desirable safety guarantees; we contrast these with the
weaker model of \emph{causal} consistency. Then we turn our attention
to the so-called CAP theorem \cite{2000brewerCAP}
\cite{2002gilbertlynchCAP}, which captures a fundamental
consistency/availability (C/A) tradeoff in the presense of network
partitions (P). We observe that the CAP theorem effectively prohibits
strong consistency in real-world scenarious, raising the question of
how we can offer safety guarantees in light of this fundamental
limitation.

Informed by the previous discussion, Section \ref{sec:des} offers a
list of three desiderata of distributed applications in the contexts
under consideration. These have been selected as especially desirable
and relevant for our use cases, and they provide a basis for assessing
the applicability of frameworks and techniques to overcome the
apparent limitations imposed by the CAP theorem.

Section \ref{sec:contcons} explains how the framework of \emph{conits}
\cite{2002tact} may be used for quantifying the nature of the C/A
tradeoff. Following Yu and Vahdat, we summarize how conit theory can
be used enforce consistency up to some real-valued $\epsilon \geq
0$. More precisely, we discuss three different approaches for
measuring the divergence of data replicas. The semantics of conits are
defined by applications, while the framework we lay out provides
general-purpose mechanisms for enforcing conit consistency. At the
extreme, conits can enforce strong consistency by setting $\epsilon =
0$. For $\epsilon > 0$, the conits framework offers neither
CAP-consistency nor CAP-availability. In return, applications provide
limited amounts of availability, possibly during network partitions,
while strictly bounding levels of inconsistency. One use case for this
is to ``smooth out'' intermittent fluctuations in network performance,
a desirable feature for safety-critical systems operating without
strict assumptions about the network.

Section \ref{sec:sheaf} is an introduction to applied sheaf theory,
which provides a highly general framework for measuring the mutual
consistency of ``overlapping'' observations (i.e. ones which we expect
to be correlated if not equal, such as the data generated by a sensor
network). We discuss an simulated example, due to \cite{}, where
sheaves are used to integrate heterogeneous sensor data, thereby
improving an estimated location for a crashed aircraft. Unlike many
introductions to the subject, we emphasize sheaves as
``topologically-flavored'' presheaves, viewing presheaves as highly
generalized transition systems. Our expectation is that this approach
makes the subject more accessible to computer scientists and serves to
highlight themes common to Sections \ref{sec:contcons} and
\ref{sec:sheaf}. It may even indicate the possibility of re-grounding
conit theory in the principled mathematical framework of sheaf theory.

Section \ref{sec:conclusion} concludes and with a ``big picture''
perspective and some suggestions for future work.

%We observe that, at
%least at face value, conit theory and sheaf theory evidently have
%little in common. However, a closer examination reveals that both
%frameworks define a notion of ``relaxed'' consistency without
%sacrificing the ability to provide quantifiable safety guarantees. It
%is no coincidence that both frameworks place particular emphasis on
%the mathematical concept of \emph{continuity}, a somewhat unusual
%characteristic in the context of distributed systems. More deeply
%exploring the connection between the sheaf-theoretic approach to
%consistency with the framework of conits may provide interesting
%avenues for future research.
