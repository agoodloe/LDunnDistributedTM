\section{Introduction}
Civil aviation has primarily been focused on efficiently moving goods
and people via the airspace to their destination. The application of
sound engineering practices and conservative operating procedures has
made flying the safest mode of transport. Now the desire not to
compromise this safety makes it challenging to introduce uncrewed
vehicles into the airspace. The rules for operating in the US national
airspace are typically relaxed during natural disasters and relief
efforts, however. The NASA Aeronautics' Airspace Operations and Safety
Program (AOSP) System Wide Safety (SWS) project has been investigating
how crewed and uncrewed aircraft may safely operate in shared
airspace, taking wildfire fighting and hurricase relief as use cases.

Disaster response scenarios like the ones under consideration present
a number of unique challenges for safe operations. Traditionally,
civil aviation has employed simple communication patterns between air
and ground and among aircraft. For instance, a typical airborne
communication protocol is Automatic Dependent Surveillance-Broadcast
(ADS-B), by which aircraft monitor and periodically broadcast their
position information to air traffic controllers and nearby
aircraft. The use cases under consideration demand more sophisticated
coordination between airborne and ground-based elements in order to
accomplish shared mission goals such as operating safely, delivering
resources, and fighting fires. However, the operating environment will
not admit the use of reliable, high-bandwidth internet connections
that consistently allow clients to transmit large amounts of data
quickly.

The need for sophisticated coordination is in tension with the fact
that the communications environment is inevitably somewhat chaotic and
unpredictable. Rather than a highly centralized topology like that of
a typical internet local area network (LAN), a more workable
communications model would be something like an ad-hoc wide area mesh
network using geographic routing \cite{}, say one supported by the
deployment of portable communication towers. Due to its ad-hoc nature,
and the particulars of our use cases, a robust communications
framework cannot rely on many hypotheses the performance of such a
network. For instance, issues like distance, terrain, smoke, and
weather mean we should expect network packets to be dropped or delayed
in unpredictable ways. These characteristics make it difficult to
coordinate distributed agents and offer strong safety guarantees in
general.

Coordinating agents over an unreliable communications medium is a
problem of distributed computing, a subdiscipline of computer
science. This purpose of this memorandum is to enumerate some of the
considerations involved in coordinating air- and ground-based elements
from a distributed computing perspective, identifying challenges,
potential requirements, and frameworks that may aid in developing
solutions.

The rest is laid out as follows. Section 2 provides a high-level,
informal introduction to the challenges of distributed computing. We
especially focus on Brewer's CAP theorem, highlighting the well-known
consistency/availability (C/A) tradeoff in the presense of network
partitions. We also offer a list of three desiderata we might expect
of distributed applications in the sort of contexts under
consideration.  Section 3 explains how the framework of \emph{conits}
\cite{} may be used for quantifying the nature of the C/A tradeoff in
the context of data replication, a desirable feature for
safety-critical systems. Section 4 is an introduction to applied sheaf
theory, which provides a highly general framework for measuring the
mutual consistency of ``overlapping'' observations (i.e. ones which we
expect to be correlated if not equal, such as the data generated by a
sensor network). We discuss an simulated example, due to \cite{},
where sheaves are used to integrate heterogeneous sensor data, thereby
improving an estimated location for a crashed aircraft. Section 5 is a
conclusion with suggestions for future work.
