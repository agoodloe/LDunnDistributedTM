\section{Introduction}
Civil aviation has primarily been focused on efficiently moving goods
and people via the airspace to their destination. The application of
sound engineering practices and conservative operating procedures has
made flying the safest mode of transport. Now the desire not to
compromise this safety makes it challenging to introduce uncrewed
vehicles into the airspace. The rules for operating in the US national
airspace are typically relaxed during natural disasters and relief
efforts, however. The NASA Aeronautics' Airspace Operations and Safety
Program (AOSP) System Wide Safety (SWS) project has been investigating
how crewed and uncrewed aircraft may safely operate in shared
airspace, taking wildfire fighting and hurricane relief as use cases.

Disaster response scenarios present unique challenges for safe
operations, particularly because of the unpredictable communications
environment combined with the need for system-wide
coordination. Traditionally, civil aviation has employed simple
communication patterns between air and ground and among
aircraft. Typical examples of protocols include Automatic Dependent
Surveillance-Broadcast (ADS-B), by which aircraft periodically
broadcast their position and velocity to air traffic controllers and
nearby aircraft. The use cases under consideration demand more
sophisticated coordination between airborne and ground-based elements
to accomplish mission goals such as navigating safely in close
proximity, delivering resources, and fighting fires. However, the
operating environment will not admit the use of reliable,
high-throughput internet connections that consistently allow clients to
transmit large amounts of data. For instance, obstructions
like distance, terrain, smoke, and weather mean we should expect
network packets to be dropped or delayed in unpredictable
ways. Therefore, network performance will be relatively difficult to
predict and control in these settings.

An unreliable communications environment makes it difficult to
coordinate agents and offer strong safety guarantees. This is a
challenge of distributed computing, a subdiscipline of computer
science. This purpose of this memorandum is to enumerate some of the
considerations involved in coordinating air- and ground-based elements
from a distributed computing perspective, identifying challenges,
potential requirements, and frameworks that may aid in developing
solutions.

The rest of this document is laid out as follows. Section
\ref{sec:distrsys} provides a high-level introduction to the topic of
\emph{consistency} of distributed replicas. We focus on the CAP
theorem \cite{2000brewerCAP} \cite{2002gilbertlynchCAP}, which
captures a fundamental consistency/availability (C/A) tradeoff in the
presense of network partitions (P). Section \ref{sec:des} offers a
list of three desiderata of distributed applications in the contexts
under consideration.  Section \ref{sec:contcons} explains how the
framework of \emph{conits} \cite{2002tact} may be used for quantifying
the nature of the C/A tradeoff in the context of data replication, a
desirable feature for safety-critical systems. Section \ref{sec:sheaf}
is an introduction to applied sheaf theory, which provides a highly
general framework for measuring the mutual consistency of
``overlapping'' observations (i.e. ones which we expect to be
correlated if not equal, such as the data generated by a sensor
network). We discuss an simulated example, due to \cite{}, where
sheaves are used to integrate heterogeneous sensor data, thereby
improving an estimated location for a crashed aircraft. We conclude in
Section \ref{sec:conclusion}.
