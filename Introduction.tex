\section{Introduction}

Civil aviation has primarily been focused on efficiently moving goods
and people via the airspace to their destination. The application of
sound engineering practices and conservative operating procedures has
made flying the safest mode of transportation. The desire not to
compromise such a safe system has made it difficult to introduce
unmanned and autonomous vehicles into the airspace. The rules for
operating aircraft in the US national airspace are typically relaxed
when needed to support natural disaster relief. The NASA Aeronautics'
Airspace Operations and Safety Program (AOSP) System Wide Safety (SWS)
project has been investigating how unmanned and manned aircraft may
safely operate in shared airspace using wildfire fighting and
hurricane relief as case studies.

Traditionally, civil aviation has employed simple communication
patterns between air and ground and among aircraft. The typical
airborne communication is very simple: aircraft broadcast their
position information via automatic dependent surveillance-broadcast
(ADS-B). Air-ground communication basically consists of radio
communication between pilots and air traffic controllers. The use
cases under consideration demand more sophisticated coordination
between airborne and ground-based elements in order to accomplish
shared mission goals such as operating safely, delivering resources,
and fighting fires. Yet the operating environment will not admit the
use of reliable high-bandwidth internet connections that enable
reliable transfers of large data. At best, we can assume some form of
ad-hoc wireless mesh network with performance characteristics quite
unlike that of modern household internet connections, with messages
being sporadically dropped or delayed due to ad-hoc routing
protocols. Coordinating distributed agents over an unreliable
communications medium necessitates a sophisticated analysis of the
situation, which is the subject of this memorandum.

\subsection{Distributed systems}
A distributed system, broadly construed, is a collection of
independent entities that cooperate to solve a problem that cannot be
individually solved \cite{kshemkalyani_singhal_2008}. This memo is
concerned with the sorts of distributed systems involved in civil
aviation, particularly in the context of disaster response scenarios
like combating wildfires and providing hurricane relief. In this
context, entities such as emergency responders, firefighters, crewed
and uncrewed aircraft teams, air traffic controllers, and other
personnel cooperate to solve shared tasks like navigating safely,
delivering resources, extinguishing fires, and collecting and sharing
data. This memo analyzes the situation from the perspective of
distributed computing. Singhal and Shivaratri \cite{10.5555/562065} characterized a
distributed computing systems as
\begin{quotation}
   A collection of computers that do not share common memory or a common
   physical clock, that communicate by a messages passing over a communication
   network, and where each computer has its own memory and runs its own
   operating system.
\end{quotation}
Our basic model is that ground-based and airborne agents interact as
\emph{clients} to computers that are connected by some type of
communications network. This network may be multi-modal, such as an
assemblage of two or more different radio systems with some kind of
bridge for passing messages between them. In general, it will be far
less reliable and achieve much less throughput than, for example, a
fiber-optic network located in a modern datacenter. Computers should
be understood in the broadest sense, taken to include aircraft systems
like ADS-B, portable communications equipment, satellites,
vehicle-mounted personal computers, or smart phones carried by
firefighters in the field.

Absence of a common memory implies communication takes place by
passing messages over the network, as opposed to writing shared data
to a common memory location. Essentially the defining characteristic
of distributed computing is that the network is not perfectly reliable
and communication over it is not instantaneous. This is particularly
true during emergency response scenarios where high-bandwidth
communication is not consistently available to all clients. For
example, two firefighting ground teams may not be able to directly
communicate over a radio network if the teams are separated by a tall
ridgeline. Unreliability of a shared communication medium brings with
it inherent challenges. Handling these sorts of challenges
gracefully---for example, passing messages over the ridgeline using a
nearby drone to route messages between the two parties---requires more
sophisticated coordination protocols than have traditionally been
applied in this domain. The inherent imperfections of the network
imply that distributed applications are subject to theoretical and
practical limitations. This memo discusses some of these limitations
and indicates how they may be handled gracefully, without sacrificing
the high safety standards of aviation.

The basic challenge of a distributed computing system is to provide
the illusion that the independent computers actually act ``as a single
coherent computer'' \cite{TanenbaumSteen07}. In other words, clients
interacting with their computers should be presented with the illusion
that they are in fact accessing a single, central computer which is
simultaneously aware of every other client. This boils down to
shielding users from the assumed limitations of the communications
network. For a variety of reasons, real-world systems will fall short
of behaving as a single perfectly coherent system. One reason for this
is that there is a deep and well-understood tradeoff between system
coherence and system performance---enforcing consistency comes at the
cost of additional communications, which imposes a performance
overhead since this process is not instantaneous. Lack of instantenous
communications requires system nodes to spend time blocking (that is,
sending messages to or waiting for responses from remote system nodes,
rather than handling requests from local clients). Another reason for
imperfect system coherence is that, besides being slow, the network
may be unreliable, even exhibiting behavior that can look like that of
an intentional adversary. For example, it may fail to deliver some
messages, or delay them unpredictably, or rearrange the order in which
they are delivered. Such imperfect network behavior presents obstacles
to consistency, particularly if the system should also exhibit good
performance.

For all of the preceding reasons, it is desirable to identify
frameworks for designing and reasoning about distributed applications
and protocols that can cope with network imperfections without
compromising safety. This memo considers three high-level desiderata
that we might expect from such frameworks, explaining the relevance of
each one to the setting under consideration, i.e. civil aviation and
disaster response. Then we consider two topics from different parts of
the distributed systems literature that are relevant for building
distributed applications, evaluating both of them with respect to our
identified desiderata. These two topics are essentially independent,
one being a general-purpose replication mechanism and the other being
an application-level framework for combining data from multiple
sources. What these topics have in common is an ability to adapt to
the challenges of the settings under consideration while providing
quantifiable bounds for system (in)consistency.
